\documentclass[fontset=windows]{article}
\usepackage{anyfontsize}
\usepackage[UTF8]{ctex}
\usepackage[english]{babel}
\usepackage[letterpaper,top=2cm,bottom=2cm,left=3cm,right=3cm,marginparwidth=1.75cm]{geometry}
\usepackage{amsmath}
\usepackage{graphicx}
\usepackage{floatrow}
\usepackage[colorlinks=true, allcolors=blue]{hyperref}

\title{\textbf{数据科学导论实验报告}}
\author{吕思翰\ 来泽远\ 曹宸瑞}

\begin{document}
\renewcommand{\figurename}{图}
\maketitle

% \tableofcontents
% \newpage

% \ctexset{abstractname=摘要}
% \begin{abstract}

% \end{abstract}

\section{比赛名称}

Predict CO2 Emissions in Rwanda

\section{成员}

\begin{itemize}
      \item 吕思翰 PB21000144
      \item 来泽远 PB21000164
      \item 曹宸瑞 PB21020659
\end{itemize}

\section{问题定义}

\subsection{Predicting CO2 Emissions}

准确监测碳排放能力是应对气候变化的重要步骤。精确的碳排放数据使研究人员和政府能够了解碳排放的来源和模式。尽管欧洲和北美已经建立了广泛的地面碳排放监测系统,但在非洲可用的系统相对较少。本任务要求参赛选手依据过往二氧化碳排放数据预测未来的排放数据。

\subsection{Dataset}

从卢旺达多个地区挑选了大约497个独特的地点,分布在农田、城市和发电厂周围。这次比赛的数据是按时间划分的;训练数据中包含2019 - 2021年的二氧化碳排放数据,任务是预测2022年至11月的二氧化碳排放数据。

\subsection{Evaluation}

本次比赛的评价指标是均方根误差(RMSE)。RMSE是预测值和实际值之间差异的平方的平均值的平方根。RMSE的值越低,表示模型的预测能力越好。RMSE定义为:$$RMSE=\sqrt{\frac{1}{N}\sum\limits_{i=1}^{N}(y_i-\hat y_i)^2}$$其中$y_i$是真实值,$\hat y_i$是预测值,N是样本数量。

\section{做题思路}

\subsection{数据概览与特性}

\begin{table}
      \centering
      \begin{tabular}{|l|l|l|l|}
      \hline
          \# & Column & Non-Null Count & Dtype \\ \hline
          0 & ID\_LAT\_LON\_YEAR\_WEEK & 79023 non-null & object \\ \hline
          1 & latitude & 79023 non-null & float64 \\ \hline
          2 & longitude & 79023 non-null & float64 \\ \hline
          3 & year & 79023 non-null & int64 \\ \hline
          4 & week\_no & 79023 non-null & int64 \\ \hline
          5 & SulphurDioxide\_SO2\_column\_number\_density & 64414 non-null & float64 \\ \hline
          6 & SulphurDioxide\_SO2\_column\_number\_density\_amf & 64414 non-null & float64 \\ \hline
          ... & ~ & ~ & ~ \\ \hline
          74 & Cloud\_solar\_zenith\_angle & 78539 non-null & float64 \\ \hline
          75 & emission & 79023 non-null & float64 \\ \hline
      \end{tabular}
\end{table}

\begin{itemize}
      \item 数据集共有79023行,76列,其中75列为特征,1列为预测值。
      \item 除经纬度外其余值均为`float64`,不需要做额外数据处理。
      \item 部分测试量包含空值,可以进行特殊值处理。
      \item 可用作索引的值有`latitude longitude year week\_no`,需预测值为`emission`,其余特征
\end{itemize}

\subsection{特征分析}
\subsection{数据预处理}
\subsection{模型训练}

\section{评估}
\section{团队成员分工}

\subsection{吕思翰}

\subsection{来泽远}

\subsection{曹宸瑞}

\section{个人总结与感悟}

\ctexset{bibname=参考文献}
\begin{thebibliography}{100}  
\bibitem{ref1}\href{https://www.kaggle.com/code/ambrosm/pss3e20-eda-which-makes-sense}{PSS3E20 EDA which makes sense}
\bibitem{ref2}\href{https://www.kaggle.com/code/kacperrabczewski/rwanda-co2-step-by-step-guide}{Rwanda CO2: Step by step guide}
\end{thebibliography}

\end{document}
