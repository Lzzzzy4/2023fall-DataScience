\documentclass[fontset=windows]{article}
\usepackage{anyfontsize}
\usepackage[UTF8]{ctex}
\usepackage[english]{babel}
\usepackage[letterpaper,top=2cm,bottom=2cm,left=3cm,right=3cm,marginparwidth=1.75cm]{geometry}
\usepackage{amsmath}
\usepackage{graphicx}
\usepackage{floatrow}
\usepackage[colorlinks=true, allcolors=blue]{hyperref}
\usepackage{enumitem}
\setlist[itemize]{noitemsep}

\title{\textbf{数据科学导论实验报告}}
\author{吕思翰\ 来泽远\ 曹宸瑞}

\begin{document}
\renewcommand{\figurename}{图}
\renewcommand{\tablename}{表}

\maketitle

% \tableofcontents
% \newpage

% \ctexset{abstractname=摘要}
% \begin{abstract}

% \end{abstract}

\section{比赛名称}

Predict CO2 Emissions in Rwanda(预测卢旺达二氧化碳排放量)

\section{成员}

\begin{itemize}
      \item 吕思翰 PB21000144
      \item 来泽远 PB21000164
      \item 曹宸瑞 PB21020659
\end{itemize}

\section{问题定义}

\subsection{Predicting CO2 Emissions}

准确监测碳排放能力是应对气候变化的重要步骤。精确的碳排放数据使研究人员和政府能够了解碳排放的来源和模式。尽管欧洲和北美已经建立了广泛的地面碳排放监测系统,但在非洲可用的系统相对较少。本任务要求参赛选手依据过往二氧化碳排放数据预测未来的排放数据。

\subsection{Dataset}

从卢旺达多个地区挑选了大约497个独特的地点,分布在农田、城市和发电厂周围。这次比赛的数据是按时间划分的;训练数据中包含2019 - 2021年的二氧化碳排放数据,任务是预测2022年至11月的二氧化碳排放数据。

\subsection{Evaluation}

本次比赛的评价指标是均方根误差(RMSE)。RMSE是预测值和实际值之间差异的平方的平均值的平方根。RMSE的值越低,表示模型的预测能力越好。RMSE定义为:$$RMSE=\sqrt{\frac{1}{N}\sum\limits_{i=1}^{N}(y_i-\hat y_i)^2}$$其中$y_i$是真实值,$\hat y_i$是预测值,N是样本数量。

\section{做题思路}

\subsection{数据概览与特性}

\begin{table}[h]
      \centering
      \begin{tabular}{|l|l|l|l|}
      \hline
          \# & Column & Non-Null Count & Dtype \\ \hline
          0 & ID\_LAT\_LON\_YEAR\_WEEK & 79023 non-null & object \\ \hline
          1 & latitude & 79023 non-null & float64 \\ \hline
          2 & longitude & 79023 non-null & float64 \\ \hline
          3 & year & 79023 non-null & int64 \\ \hline
          4 & week\_no & 79023 non-null & int64 \\ \hline
          5 & SulphurDioxide\_SO2\_column\_number\_density & 64414 non-null & float64 \\ \hline
          6 & SulphurDioxide\_SO2\_column\_number\_density\_amf & 64414 non-null & float64 \\ \hline
          ... & ~ & ~ & ~ \\ \hline
          74 & Cloud\_solar\_zenith\_angle & 78539 non-null & float64 \\ \hline
          75 & emission & 79023 non-null & float64 \\ \hline
      \end{tabular}
      \caption{数据集概览}
\end{table}

\begin{itemize}
      \item 数据集共有79023行,76列,其中75列为特征,1列为预测值。
      \item 除经纬度外其余值均为`float64',不需要做额外数据处理。
      \item 部分测试量包含空值,可以进行特殊值处理。
      \item 可用作索引的值有`latitude longitude year week\_no',需预测值为`emission',其余特征
\end{itemize}

\subsection{特征分析}

\subsubsection{经纬度}

\subsubsection{年份}

\subsubsection{周数}

\subsubsection{其他特征}

除了上述特征外,数据集还给出了大量的气象数据,包括二氧化硫、一氧化碳、二氧化氮、甲醛、臭氧、紫外线气溶胶、云等物质的方位角、天顶角、深度、角度、密度、压力、温度、反射率等信息。我们认为这些信息对于预测二氧化碳排放量有一定的参考价值。以下是对这些特征的分析。

以SulphurDioxide\_SO2\_column\_number\_density为例,计算其与emission的相关系数并作出散点图,得到结果如下:

\begin{center}
Pearson correlation: -0.013960940599134839

Spearman correlation: -0.06904123013729443
\end{center}


\begin{figure}[H]
      \centering
      \includegraphics[width=1\textwidth]{output1.png}
      \caption{SulphurDioxide\_SO2\_column\_number\_density与emission的散点图}
\end{figure}

可以看出,这两者之间的相关性非常小,因此我们认为这些气象数据对于预测二氧化碳排放量的影响可以忽略不计。

我们还对其他气象数据进行了类似的分析,得到的结果也是类似的,因此我们决定舍弃这些气象数据。

\subsection{数据预处理}

因此,我们借用排放量地图进行分析,对排放量最多的地点测量得到的数据进行相关性分析。有理由相信如果CO2主要来自工厂,会同时产生大量SO2、CO等气体。以2021年CarbonMonoxide CO column number density为例,得到结果如下:

\begin{center}
      Pearson correlation: -0.0016211411671317282

      Spearman correlation: 0.002390506275078972
\end{center}

\begin{figure}[H]
      \centering
      \includegraphics[width=0.8\textwidth]{output7.png}
      \caption{\texttt{CarbonMonoxide\_CO\_column\_number\_density} 与 \texttt{emission} 的散点图}
\end{figure}

我们无法得出该气体与CO2排放量之间有明显的相关性。类似的,我们对该地点的其他气象数据进行了计算,也未能分析出得到更多有价值的信息。

Ozone\_solar\_azimuth\_angle

\subsubsection{权重量化}

\subsection{模型选择}

这是一个回归问题,我们尝试了多种回归模型,包括线性回归、逻辑回归、支持向量回归、决策树回归、类别提升、自适应提升、极端梯度提升、随机森林、半径近邻、K近邻等。

\subsubsection{LinearRegression(线性回归)}

\subsubsection{SupportVectorRegressor(支持向量回归)}

\subsubsection{DecisionTreeRegressor(决策树回归)}

\subsubsection{CatBoostRegressor(类别提升)}

\subsubsection{AdaBoostRegressor(自适应提升)}

\subsubsection{XGBoostRegressor(极端梯度提升)}

\subsubsection{RandomForestRegressor(随机森林)}

我们使用了经纬度、年份、周数作为输入,参数设置min\_samples\_leaf=6,使用随机森林模型进行训练,得到的结果如下:

\subsubsection{RadiusNeighborsRegressor(半径近邻)}

\subsubsection{KNeighborsRegressor(K近邻)}

% \subsubsection{LGBMRegressor(LightGBM回归)}

\subsection{结果优化}

显然,预测出的CO2排放量不可能为负数,因此我们将预测出的负数值置为0。

除此之外,我们相信训练集的数据中,如果某个地点在2019-2021年的CO2排放量均为0,那么这个地点在2022年的CO2排放量也应该为0。因此,我们将这些地点的预测值置为0。

代码如下:

\begin{lstlisting}[style=Python]
      y_pred[y_pred < 0] = 0
      
      zero_emissions = train.groupby(['latitude', 'longitude'])['emission'].mean().to_frame()
      zero_emissions = zero_emissions[zero_emissions['emission'] == 0]
      mask = test.apply(lambda x: (x['latitude'], x['longitude']) in zero_emissions.index, axis=1)
      y_pred.loc[mask, "emission"] = 0
\end{lstlisting}

处理后对评估结果(RMSE)有一定的提升。

\section{评估}

\subsection{RMSE}

RMSE(Root Mean Square Error,均方根误差)是一种常用的评估回归模型性能的指标。它是观测值与真实值偏差的平方和的平均值的平方根。RMSE 的值越小,说明模型的预测性能越好。在 Python 中,可以使用 scikit-learn 库的 `mean\_squared\_error` 函数计算 RMSE,代码如下:

\begin{lstlisting}[style=Python]
      def mean_squared_error(self):
      rmse = mean_squared_error(
          self.test["emission"], self.ans["emission"], squared=False
      )
      return rmse
\end{lstlisting}

\subsection{交叉验证}

TODO

交叉验证是一种评估模型性能的统计学方法。它的基本思想是将原始数据集分成两部分(训练集和验证集),然后在训练集上训练模型,在验证集上验证模型的性能。这个过程重复多次,每次都用不同的训练集和验证集,最后取平均值作为模型的最终性能。交叉验证的主要优点是它可以利用有限的数据获得模型在未知数据上的性能的更准确的估计。

在 Python 中,可以使用 scikit-learn 库的 cross\_val\_score 函数进行交叉验证,代码如下:

\begin{lstlisting}[style=Python]
      from sklearn.model_selection import cross_val_score
      scores = cross_val_score(model, X, y, cv=5, scoring='neg_mean_squared_error')
      print("RMSE: %0.2f (+/- %0.2f)" % (-scores.mean(), scores.std() * 2))
\end{lstlisting}

\subsection{预测结果可视化}

\subsection{最终得分}



\section{团队成员分工}

\subsection{吕思翰}

\subsection{来泽远}

\subsection{曹宸瑞}

\section{个人总结与感悟}

\ctexset{bibname=参考文献}
\begin{thebibliography}{100}  
\bibitem{ref1}\href{https://www.kaggle.com/code/ambrosm/pss3e20-eda-which-makes-sense}{PSS3E20 EDA which makes sense}
\bibitem{ref2}\href{https://www.kaggle.com/code/kacperrabczewski/rwanda-co2-step-by-step-guide}{Rwanda CO2: Step by step guide}
\end{thebibliography}

\end{document}
